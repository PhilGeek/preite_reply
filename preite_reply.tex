%!TEX TS-program = xelatex 
%!TEX TS-options = -synctex=1 -output-driver="xdvipdfmx -q -E"
%!TEX encoding = UTF-8 Unicode
%
%  preite_reply
%
%  Created by Mark Eli Kalderon on 2019-05-08.
%  Copyright (c) 2019. All rights reserved.
%

\documentclass[12pt]{article} 

% Definitions
% \newcommand{\change}{\textcolor{blue}{\textbf{CHANGE SLIDE}}}
\newcommand\mykeywords{Plato, Timaeus, World-Soul, cognition}
\newcommand\myauthor{Mark Eli Kalderon}

% Packages
\usepackage{geometry} \geometry{a4paper} 
\usepackage{url}
% \usepackage{txfonts}
\usepackage{color}
\usepackage{enumerate}
\definecolor{gray}{rgb}{0.459,0.438,0.471}
\usepackage{setspace}
% \doublespace % Uncomment for doublespacing if necessary
% \usepackage{epigraph} % optional

% XeTeX
\usepackage[cm-default]{fontspec}
\usepackage{xltxtra,xunicode}
\defaultfontfeatures{Scale=MatchLowercase,Mapping=tex-text}
\setmainfont{Hoefler Text}
\newfontfamily{\sbl}{SBL Greek}

% Bibliography
% \usepackage[round]{natbib}

% Title Information
\title{Reply to Preite}
\author{\myauthor}
\date{} % Leave blank for no date, comment out for most recent date

% PDF Stuff
\usepackage[plainpages=false, pdfpagelabels, bookmarksnumbered, backref, pdftitle={Perception and Extramission in De quantitate animae}, pdfkeywords={\mykeywords}, xetex, colorlinks=true, citecolor=gray, linkcolor=gray, urlcolor=gray, unicode=true]{hyperref} 

%%% BEGIN DOCUMENT
\begin{document}

% Title Page
\maketitle
% \begin{abstract}
%
% \end{abstract}

% Layout Settings
\setlength{\parindent}{1em}

% Main Content

\section{Contact} % (fold)
\label{sec:contact}

I have a query about how to read \emph{ephaptetai}. Preite claims that it literally means contact when applied to divisible beings but means something else when applied to indivisible beings since these do not admit of corporeal contact. While there is nothing incoherent \emph{per se} with reading an occurence of the verb as corporeal contact, I doubt that reading coheres with the text. Specifically, the World-Soul is said to be in ``contact'' with divisible and indivisible beings. So whatever sense is assigned to the verb it should intelligibly apply with that sense to both. But interpreted as corporeal contact, the verb does not intelligibly apply to indivisible beings. So the verb, in that grammatical context, must receive a general reading so that it may intelligibly apply to both divisible and indivisible beings.

% section contact (end)

\section{First Embodiment} % (fold)
\label{sec:first_embodiment}

It is unclear to me how the First Embodiment interpretation of T2 could be read as a thought experiment in the way suggested. The souls of mortal beings are composed of indivisible and divisible forms of Being, Sameness, and Difference, if less pure. Indeed, of a second or third degree. So there is a substantial difference between the souls of mortal beings and the World-Soul since they differ in the purity of the material from which they are generated. If you think that the substance from which the World-Soul is composed explains, even in part, aspects of its cognitive activity, then the substantial difference potentially limits the value of any such thought experiment. If there is a difference in cognitive activity between the World-Soul and the soul of a mortal being embodied and embedded in an environment with strong powers how are we to know if this is due to the mortal being embodied and embedded in an environment or if this is due to the impurity of its substance? 

Not only is there a substantial difference between the World-Soul and the souls of mortal beings, but the manner in which these souls are united with their bodies differs as well. The World-Soul encompasses the sensible and the corporeal. The souls of mortal beings, by contrast, are encompassed by their bodies. Bound to the marrow that constitues the brain, encased in hard bone by the skull and padded out with flesh and hair, the immortal part of the soul resides within the human head where it reigns over the mortal parts of the soul and the rest of the body. Here too it is fair to wonder whether the manner in which the soul is united to the body makes a difference to the cognition of the sensible and the corporeal. The subject matter of mortal opinion is exogenous. The subject matter of the cosmic Living Being's opinion is endogenous. Might this not make a difference to their cognition of the sensible and the corporeal? While not unconnected with the fact that mortal beings are embodied and embedded in an environment, this specific difference seems to have more to do with a difference in the soul--body union.

The souls of mortal beings are not just like the World-Soul, their elder sister. They are composed of a less pure substance, and are united in a different manner to the body they animate. And so we cannot automatically attribute any difference in cognitive activity to the fact that mortal beings and embodied and embedded in an environment in the way that the World-Soul is not, thus limiting the value of Timaeus' narrative of first embodiment as a thought experiment that will reveal the cognitive effects of being embodied and embedded in an environment with strong powers.

% section first_embodiment (end)

\section{Like Is Known by Like} % (fold)
\label{sec:the_objects_of_the_world_soul_s_cognition}

Preite wants to resist Crocilius's suggestion that the object of the World-Soul's cognition is restricted to encosmic corporeal beings with divisible and indivisible aspects. Instead, reading this passage in line with the proemium, she wants to suggest that the World-Soul may cognize both intelligible and sensible objects. I too am sympathetic with this more traditional reading and am happy enough with what she has to say in favor of here reading. However, there is an aspect of Corcilius' discussion that has been overlooked that bears on the applicability of the principle that like is known by like. The most straightforward application of this principle is where a perceiver may cognize an object because of a substantial similarity between them. Someone may perceive fire by being composed, at least in part, by fire. Corcilius observes that while the World-Soul may be composed of intermediate forms of Being, Sameness, and Difference, the corporeal and sensible are not. Theaetetus may have being, but he is not composed of it. I think that Corcilius has a pont here. This may not require that we withhold attributing the principle that like is known by like to Timaeus, but if we do, it requires that we, at the very least qualify it, since there is no substantial likeness as would be required by the principle in its unqualified form. 

% section the_objects_of_the_world_soul_s_cognition (end)

\section{Cognitive Failure} % (fold)
\label{sec:cogntive_failure}

I also had some queries about Cognitive Failure. Specifically, I wonder whether there may be evidence of a more profound cognitive failure earlier on in the passage. Begin with an apparent inconsistency that has long bothered me. T2 begins with the claim that the revolutions of the soul, when plunged into the violent stream of nutriment and sensation, neither master the stream nor are mastered by it. Later on, in the line following the passage Preite labels as Cognitive Failure, Timaeus claims that the revolutions of the soul seem to master the stream but in fact is mastered by it. How are we to make sense of this apparent inconsistency? Taylor makes the following suggestion. Timaeus is narrating the deleterious effects of shock of embodiment and the process of recovery. There is no inconsistency if the apparently inconsistent claims are made about different phases in the soul's recovery from the shock of embodiment. In an earlier phase, the revolutions of the soul neither master nor are mastered by the stream of nutriment and sensation. At a later phase, the revolutions of the soul are mastered by the stream despite an illusion of mastery. Moreover, Taylor at least, links these different phases with different cognitive impairments. In the earlier phase, when the revolutions neither master nor as mastered, the newly incarnate mortal being is not yet capable of judgment. In the later phase, when the revolutions are mastered though they appear to master, the newly incarnate mortal being is capable of judgment but only of false judgment. Presumably, on Taylor's reading then, true judgment only emerges when the revolutions of the soul master the stream. I don't want to defend Taylor per se, only to use Taylor's interpretation to suggest that there is scope for attributing a more profound cognitive failure than making systematically false judgments.

% section cogntive_failure (end)

\section{Sticking up for Brisson} % (fold)
\label{sec:sticking_up_for_brisson}

One of the commendable things about Preite's paper is her careful review of different models of mortal cognition and whether or not they can be made to cohere with the comparative model allegedly put forward in T1. This is important work. And since it is, I would like to gently push back against the criticism's directed toward Brisson. Before coming to that, let's consider one worry that was raised about applying the comparative model to mortal cognition, namely, that not all mortal judgments are explicitly comparative. We may judge not only that Socrates is similar to Simmias but also that Socrates is pug-nosed. How are one place predications understood on the comparative model. This, of course, depends on how one construes the comparative model. Experimenting, one might try to say that Socrates is the same as the pug-nosed, where the ``pug-nosed'' divides its reference among the plurality of pug-nosed beings. In saying that Socrates is pug-nosed one says that Socrates is the same as the pug-nosed in the sense of being like them in being, well, pug-nosed. Notice that this approaches, at least, Brisson's interpretation. On that interpretation, to judge the same as the same or the different as different is to judge truly, while judging the same as different, or the different as the same, is to judge falsely. The same divides its reference between n-tuples of things that are similar in some respect. Socrates and the pug-nosed are similar with respect to being pug-nosed. To say of the same, say Socrates and the pug-nosed, that they are in fact the same is to judge truly. To say of the different, say Socrates and the foolish, that they are in fact different is to judge truly as well. 

Brisson motivates this reading by appealing to the \emph{Sophist}, an appeal criticized by Preite. I want to set this aside. Even if Brisson is wrong about the \emph{Sophist} he still may yet be right about the \emph{Timaeus}. It is important to bear in mind that while there is an overlap in themes and language between the \emph{Sophist} and the \emph{Timaeus}, there are also important doctrinal differences. One prominent example is that the greatest kinds, Motion and Rest, have gone missing in the \emph{Timaeus}. And indeed Timaeus seems to have reverted to explaining Motion in terms of Difference and Rest in terms of Sameness thus obviating the need to postulate these kinds. So we should not expect Timaeus' views to perfectly cohere with the Eleatic Stranger's. 

I am not sure what Preite's objection to Brisson is meant to be independently of her criticism of his reading of the Sophist. Somehow, we arrive at a view that links difference with falsity and sameness with truth, but that is explicitly not Brisson's view. So I am left wondering what is, in fact, wrong with Brisson's interpretation.



% section sticking_up_for_brisson (end)


%Bibliography
% \bibliographystyle{plainnat}
% \bibliography{Philosophy}

\end{document}